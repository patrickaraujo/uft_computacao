\documentclass[12pt]{article}
\usepackage{enumerate}% http://ctan.org/pkg/enumerate
\usepackage{sbc-template}

\usepackage{graphicx,url}

\usepackage[brazil]{babel}   
%\usepackage[latin1]{inputenc}  
\usepackage[utf8]{inputenc}  
% UTF-8 encoding is recommended by ShareLaTex

     
\sloppy

\title{AS ABORDAGENS DA BITCOIN E BLOCKCHAIN EM ARTIGOS ACADÊMICOS}

\author{Patrick Anderson Matias de Araújo\inst{1}}


\address{Acadêmico do curso de Ciência da Computação da Universidade Federal do Tocantins. \\Trabalho apresentado à disciplina Computador e Sociedade, \\ministrada pelo Professor Doutor Patrick Letouzé Moreira, \\no segundo semestre de 2017.
}

\begin{document} 

\maketitle
     
\begin{resumo} 
  Diante das discussões promovidas pela disciplina Computadores e Sociedade, sobre os tópicos bitcoin e blockchain, suas empregabilidades e efeitos que transcendem a utilização prática dessas ferramentas, a necessidade de elaborar um trabalho sobre o assunto me fez estabelecer uma metodologia de consulta para a verificação, na base de dados de uma plataforma de divulgação científica, a saber, SciELO. O objetivo da metodologia contribuiu para ajudar a encontrar trabalhos que analisam a moeda virtual e a tecnologia blockchain em todos os aspectos possíveis. Diante da ocorrência de quatro artigos encontrados na plataforma, em um levantamento feito em 3 de novembro de 2017, publicados entre os anos de 2015 e 2016. Foi impossível criar categorias de abordagens pela falta de textos, o que, por outro lado, favorecerá a análise de todos os trabalhos encontrados. Os objetivos dessa verificação são, no primeiro momento, verificar quais abordagens as ferramentas recebem em textos acadêmicos; e, em segundo, situar as ferramentas nas abordagens acadêmicas veiculadas aqui no Brasil. Esses dois objetivos servirão para que eu analise de que maneira as ferramentas têm sido avaliadas pela comunidade acadêmico-cientifica e, assim, discutir possíveis justificativas para tais abordagens.
\end{resumo}

\begin{abstract}
In the face of the discussions put forward by the subject Computers and Society, about bitcoin and blockchain, and the possibilities and effects that beyond goes  the simple use of the tools, the requirement of making a college assignment impose me to establish a scientific methodology of query to examine in the database of a scientific platform, named, SciELO. The goal of the methodology was to help find papers about the virtual coin and the blockchain technology in every single way. In view of the occurrence of four papers found in the database, published between the years of 2015 and 2016, in a query realized in November 3rd, 2017. It became impossible to create categories of approaches because the lack of more texts, on the other hand, it makes easy to analyze all the papers found. The purposes of this examination were, in the first moment, examine which approaches the tools receives in academic texts; and furthermore, find the approaches that the tools receive in Brazilian academic texts. Both of the purposes will help to analyze which way the tools were analyzed by the scientific community, and so, discuss possibles justifications for such approaches.
\end{abstract}

\section{Introdução}

Abaixo, estabeleço uma lista com as ocorrências encontradas. Além do título, a tabela encontra-se disposta com nome de autores, níveis de formação acadêmica dos mesmos, instituição em que os trabalhos foram produzidos, ano de publicação e \textit{links} de acesso:

\begin{enumerate}[I]

\item Título: Análise dos benefícios sociais da bitcoin como moeda

\begin{itemize}

\item Autor: Salete Oro Boff / Natasha Alves Ferreira

\item Nível: Doutora / Mestranda

\item Instituição: Programa de Pós-Graduação Stricto Sensu em Direito IMED - Faculdade Meridional

\item Ano: 2016

\item Link: http://www.scielo.org.mx/pdf/amdi/v16/1870-4654-amdi-16-00499.pdf

\end{itemize}


\item Título: The Digital Currency Challenge for the Regulatory Regime
\begin{itemize}

\item Autor: 
Gonzalo Arias Acuña / Andrés Sánchez Pullas
\item Nível: Informação não disponível no texto.
\item Instituição: Universidad Adolfo Ibáñez / University of Melbourne
\item Ano: 
2016
\item Link: 
http://www.scielo.cl/pdf/rchdt/v5n2/0719-2584-rchdt-5-02-00173.pdf
\end{itemize}


\item Título: A MIXED BLESSING: RESILIENCE IN THE ENTREPRENEURIAL SOCIO-TECHNICAL SYSTEM OF BITCOIN
\begin{itemize}

\item Autor: 
Marcel Morisse / Claire Ingram
\item Nível: Informação não disponível no texto.
\item Instituição: University of Hamburg, Hamburg, Germany / Stockholm School of Economic, Stockhol, Sweden
\item Ano: 
2016
\item Link: 
http://www.scielo.br/pdf/jistm/v13n1/1807-1775-jistm-13-1-0003.pdf
\end{itemize}


\item Título: A few South African centsworth on bitcoin
\begin{itemize}

\item Autor: 
A. Nieman
\item Nível: Informação não disponível no texto.
\item Instituição: LLB; LLD (North-West University). Practising Advocate, Johannesburg Society of Advocates, Sandton. Part-time lecturer, Department of Auditing, University of Pretoria.
\item Ano: 
2015
\item Link: 
http://www.scielo.org.za/pdf/pelj/v18n5/26.pdf
\end{itemize}

\end{enumerate}

O período anunciado no título desse \textit{paper} diz respeito às ocorrências encontradas. De fato, esse período bianual resume a produção acadêmica sobre o assunto na atualidade. Ao chegarmos no limite final do ano de 2017, talvez as publicações periódicas guardem outros materiais a respeito.

\section{Breve análise dos artigos encontrados}

O primeiro texto disposto na plataforma, “A few South African cents worth on bitcoin”, de 2015, se configura como uma introdução do conceito de moedas virtuais no mercado financeiro. O cenário descrito é específico e desenha a situação de não utilização do bitcoin, já que o país, A África do Sul, não possui, à época, uma legislação que autorize essa utilização.

Do corpus encontrado, esse foi o único texto publicado em 2015. As demais ocorrências são de 2016. É interessante notar essa avaliação de uma moeda digital em um contexto de não legalidade porque o momento e cenário descritos desafiam a política local para o entendimento de utilização da tecnologia e, em certa medida, para o desenvolvimento de leis que o façam.

O texto se posiciona como elemento detonador dessas discussões que envolvem avanço tecnológico e mercado financeiro, além da alteração de alguns conceitos do Direito. Discutir a utilização da moeda virtual significará, na África do Sul, formular novas formas de atuação econômica. O artigo defende, então, que a utilização do \textit{bitcoin}, a mais comum dessas moedas virtuais, é positiva ao mercado financeiro sul-africano.

No fim, o texto serve como apelo para que uma legislação seja criada em torno da utilização das moedas virtuais. O texto se coloca, de maneira clara, como veículo de defesa da tecnologia.

O segundo trabalho, “The Digital Currency Challenge for the Regulatory Regime”, segue a linha de definição encontrada no primeiro texto. Assim, definir o que é uma moeda digital é o primeiro objetivo do artigo. Em seguida, os autores investigaram a funcionalidade da moeda. Em seguida, será preocupação do texto estabelecer a relação das moedas com o mercado financeiro virtual, apontado como ponto inicial nessa cadeia comercial. Por último, assim como o texto da África do Sul, o desafio do texto será tratar de uma ferramenta cuja legalidade não se encontra regulamentada.

Ao pensar nas vinculações institucionais dos dois autores, universidades do Chile e da Austrália, respectivamente, posso inferir que o problema da legalização, agora, foi expandido para mais dois continentes. Isso serve para uma conclusão de que, mesmo com o passar do tempo, a tecnologia continua desafiando os interessados numa escala global.

O texto aponta o criador do \textit{bitcoin}, Satoshi Nakamoto, como um programador talentoso. A sua invenção, descrita como maneira de ganhar de dinheiro de forma fácil, seria a prova da configuração de uma mente brilhante.

O contexto da internet facilita o desenvolvimento de moedas digitais. No entanto, mais uma vez, a dificuldade de definição desse recurso é apontada.

O texto aponta uma possibilidade comum quando se deseja definir a moeda digital. Essa possibilidade estaria ligada a uma ideia  de combinação de palavras (ciber-moedas, \textit{e-coin}, por exemplo). Essa construção liga a ideia de algo eletrônico com o dinheiro material em si. 

A segunda movimentação do texto refere-se a uma abordagem mais técnica das moedas digitais. Aponta, assim como no primeiro texto, o caráter de avanço do \textit{bitcoin} no mercado.

Por fim, trata dos papeis do Estado na regulação da utilização das moedas digitais. Esses desafios passam pela definição das melhores maneiras de um serviço utilizado, basicamente, na internet, cuja natureza dificulta tal utilização regulamentada. Outro aspecto que não fica de fora dessa discussão é a mudança do comportamento social quando a adoção de serviços mais virtuais começa dividir espaço com as alternativas mais físicas.

Mas o texto reconhece que é difícil falar de comportamento do consumidor. Porém, o escrito garante uma coisa: é difícil representar essa moeda virtual no cotidiano prático do consumidor. As representações materiais mais usuais do dinheiro, ouro, sal, açúcar, trigo ou metais de toda a natureza, ainda desafiam o entendimento dessa moeda de materialidade volátil, de impressão e representação quase impossíveis.

Assim como o primeiro texto, esse artigo encerra defendendo o caráter inovação e praticidade das moedas digitais. O potencial econômico também é apontado.

A utilização das moedas necessita de novas relações econômicas e financeiras. As conclusões não alteram muito que se afirmou no primeiro texto, mas posso entender que o contexto de afirmação está ampliado. Agora, não só um país africano, mas um sul-americano e outro da Oceania definem, e defendem, a utilização das moedas, bem como a necessidade de se legislar esse potencial tecnológico.

O terceiro texto, “A mixed blessing: resilience in the entrepreneurial socio-technical system of bitcoin”, também de 2016, mais complexo que os dois primeiros, utiliza a teoria da Resiliência, a princípio da Psicologia, para investigar a utilização do \textit{bitcoin} através de entrevistas com oito empreendedores de países europeus. Os autores, de universidades europeias, uma da Suécia e outra da Alemanha, ampliam, então, o raio de abrangência da moeda virtual porque agora o continente europeu foi inserido na discussão.

O texto está preocupado em responder as formas de atuação desses oito empreendedores diante da falência de uma grande empresa chamada Mt.Gox. A investigação buscará, por meio das entrevistas, as formas de recuperação apresentas pelos profissionais. Nesse sentido, o texto não apresenta um interesse direto para esse \textit{paper}.

Pela configuração do terceiro texto, passo ao quarto e último, de duas autoras brasileiras, intitulado, “Análise dos benefícios sociais da bitcoin como moeda”, também de 2016. Em seus objetivos, a verificação dos efeitos positivos da moeda está anunciada logo no título.

O texto aponta a relação entre tecnologia e Economia como possibilidade para solucionar problemas sociais. O Artigo sinaliza o método dedutivo (em que o raciocínio lógico é obtido via dedução) como base da metodologia. As fases da metodologia serão obtidas por meio da análise um corpus adquirido via pesquisa bibliográfica.

Assim como os dois primeiros textos, o artigo brasileiro aponta o desenvolvimento de inovações como base da qualidade de vida humana. Essas inovações envolvem técnicas agrícolas, comerciais e sociais, a exemplo do aprimoramento dos meios de transporte e de comunicação, por exemplo.

O dinheiro material aparece como uma das principais conquistas tecnológicas. O seu emprego, nesse sentido, pode ser visto como um grande avanço nas relações sociais. A sua condição de adaptação também aparece sinalizada no artigo. Essa condição se mostrou coerente, por exemplo, com o avanço da internet. Nesse caso, esse processo aparece mais destacado que nos outros dois textos.

Desse modo, o desenvolvimento de uma moeda virtual como a \textit{bitcoin} desafia tais configurações quando exige a adaptação do mercado financeiro. O que não se mostra diferente na história evolutiva das negociações econômicas. Mesmo assim, a moeda eletrônica mais conhecida não encontra, por outro lado, uma garantia de utilização em longo prazo. 

O passo futuro será, segundo o texto, a sua regulação legal. Esse processo confirmará ou não a consolidação desse tipo de moeda.

Essa discussão deverá fazer parte das pautas de consumidores, operadores financeiros e legisladores no mundo todo. Outra grande conquista advinda de tais discussões será o impacto positivo de relações mais transparentes entre governos e governados.

Um avanço do texto brasileiro é o tratamento do \textit{blockchain} como alternativa viável em meio a todo esse processo de utilização de moedas virtuais. Essa defesa não acontece de maneira mais detalhada nos outros três artigos. Trata-se, aqui, de um processo de registro de dados, tecnologia que visa à descentralização como medida de segurança. A disposição da informação em diversos servidores impede o ataque sistemático mais eficaz, o que o artigo das estudiosas brasileiras insinua nas suas considerações finais.


\section{Considerações finais}

A leitura comparativa dos quatro textos e, mais especificamente, dos três artigos que tratam especificamente de moedas virtuais, aponta para um cenário de concordância com relação à utilização do \textit{bitcoin} nos últimos dois anos. Trata-se da mais comum entre as disponíveis no mercado por conta da sua configuração e divulgação.

Sendo tratada como uma tecnologia inerente aos processos de sociabilização, a moeda material não poderia ficar de fora da revolução que a internet apresenta. Nessa configuração, o aparecimento das moedas virtuais segue essa evolução.

Em todos os textos lidos, os autores apontam que esse surgimento tecnológico impulsionam discussões no campo da Economia, Direito e Legislação estatal. No caso do texto brasileiro, encontro a defesa de que caminhamos, sim, para a legalização das moedas digitais. Somente a partir daí poderemos analisar se essa utilização se acentua ou não.

\section{Referências}

ARIAS ACUNA, Gonzalo; SANCHEZ PULLAS, Andrés. The Digital Currency Challenge for the Regulatory Regime. \textbf{Rev. chil. derecho tecnol.}  , Santiago , v. 5, n. 2, p. 173-209, dic. 2016 . Disponível em \textless http://www.scielo.cl/pdf/rchdt/v5n2/0719-2584-rchdt-5-02-00173.pdf\textgreater. Acesso em 09 nov. 2017.

\noindent BOFF, Salete Oro; FERREIRA, Natasha Alves. Análise dos benefícios sociais da bitcoin como moeda. \textbf{Anu. Mex. Der. Inter}, México, v. 16, p. 499-523, dic. 2016. Disponível em \textless http://www.scielo.org.mx/pdf/amdi/v16/1870-4654-amdi-16-00499.pdf\textgreater. Acesso em 09 nov. 2017.

\noindent MORISSE, Marcel; INGRAM, Claire. A MIXED BLESSING: RESILIENCE IN THE ENTREPRENEURIAL SOCIO-TECHNICAL SYSTEM OF BITCOIN. \textbf{ JISTEM J.Inf.Syst. Technol. Manag.},  São Paulo ,  v. 13, n. 1, p. 3-26,  Apr.  2016 .   Disponível em \textless http://www.scielo.br/pdf/jistm/v13n1/1807-1775-jistm-13-1-0003.pdf\textgreater. Acesso em  09  Nov.  2017.

\noindent NIEMAN, A. A few South African cents worth on bitcoin. \textbf{PER}, Potchefstroom ,  v. 18, n. 5, p. 1979-2010, 2015.   Disponível em: \textless http://www.scielo.org.za/pdf/pelj/v18n5/26.pdf \textgreater. Acesso em  10  Nov.  2017.

\end{document}
